\section{Storia e sviluppo dello standard CryptoNote}\label{inizio-dello-sviluppo-e-storia-dellalgoritmo}
Dato che Bitcoin fallisce dal punto di vista della privacy delle
transazioni e della resistenza agli ASICs (\emph{Application Specific
Integrated Circuits}), lo sviluppatore \textbf{Nicolas van Saberhagen},
che molti pensano essere un nome di fantasia, con alcune speculazioni
che lo associerebbero al fantomatico creatore di Bitcoin,
\textbf{Satoshi Nakamoto}, nel 12 dicembre 2012 presenta un documento
con all'interno l'idea di \textbf{CryptoNote}.

Questo innovativo protocollo di consenso viene presentato come una
possibile alternativa ai meccanismi tradizionali utilizzati dalle
criptomonete, come la \emph{Proof-Of-Work} di Bitcoin, oltretutto in grado di garantire
elevati livelli di anonimato e un'opportuna resistenza agli ASICs.
Alcune delle funzioni menzionate riguardavano transazioni di dimensione
inferiore e non facilmente associabili ad un utente, l'utilizzo delle
firme ad anello volte a migliorare la sicurezza e respingere gli
attacchi alla blockchain e l'adattamento dinamico della emissione di
moneta.

\subsection{Prime adozioni dello standard: Bytecoin e Monero}\label{sviluppo-dello-standard-cryptonight-e-adozione}
Qualche mese dopo, gli sviluppatori Seigen, Max Jameson, Tuomo Nieminen,
Neocortex e Antonio M. Juarez, pubblicano un documento, facente parte
degli standard di CryptoNote, con all'interno la descrizione della
funzione di hash per la proof-of-work di CryptoNote, chiamata
\textbf{CryptoNight}.

\textbf{Bytecoin (BCN)} è stata la prima criptomoneta ad adottare il
protocollo di consenso CryptoNote, scelta giustificata dalla volontà dei
fondatori di avere una criptomoneta volta alla privacy finanziaria,
attraverso una protezione completa dell'utente che utilizza gli
strumenti finanziari messi a disposizione, dalle transazioni
all'identità personale. Tra le altre differenze, c'è l'aggiustamento
della difficoltà di minare nuova moneta ad ogni blocco, generando un
blocco ogni due minuti circa. Nonostante le buone premesse, la moneta
oggi ha un \emph{market cap} irrisorio e non è stata adottata a causa di
svariati problemi:

\begin{itemize}
  \item Inizialmente, la moneta è stata pre-minata,
  fornendo l'80\% delle monete ad un gruppo di \emph{early adopters},
  generando una distribuzione iniqua e sleale. Bytecoin ha avuto vari
  problemi tecnici di instabilità nel corso della sua vita, con difficoltà
  da parte degli utenti che partecipavano alla rete impossibilitati a
  sincronizzare tutta la blockchain.

  \item Il 20 dicembre 2017, la rete di
  Bytecoin ha ricevuto un attacco DDoS massiccio, con lo scopo di rubare
  le monete e distribuire la potenza tra le varie monete che adottano lo
  standard CryptoNote. Gli utenti affetti erano soprattutto chi aveva un
  software non aggiornato per minare nuova moneta e chi utilizzava
  \emph{desktop wallets} e \emph{web wallets} si è visto rallentare o
  disabilitare la sincronizzazione dei pagamenti, cosa che ha aumentato lo
  sconforto dei suoi partecipanti alla rete

  \item  Non si hanno notizie
  riguardo futuri sviluppi della moneta, con l'ultimo post che prometteva
  lo sviluppo di una tecnologia per nascondere gli importi delle
  transazioni e di creare un wallet più efficiente e sicuro, risalente al
  2019.
\end{itemize}

Un'altra moneta, chiamata \textbf{Monero (XMR)}, nel aprile 2014 adottò
CryptoNote, proprio al fine di garantire la privacy e la decentralizzazione del
mining. La decisione di utilizzare questa tecnologia è stata una dei
fattori che hanno contibuito alla crescita e al
successo di Monero come una delle monete digitali più promettenti e
utilizzate. 
Monero ha avuto un ruolo cruciale nello sviluppo attivo di
CryptoNight, introducendo modifiche volte ad adattare tale funzione alle proprie necessità. 
Alcune versioni erano ottimizzate per miners con risorse limitate per il mining, altre invece sono
state rilasciate per migliorare l'efficienza e l'equità del mining,
oltre a mantenere la resistenza agli ASIC. Tuattavia, nel 2019
Monero decise di cambiare il suo algoritmo da CryptoNote a RandomX, per risolvere una serie
di problemi riscontrati in CryptoNight. \\

Il presente report inizierà fornendo una panoramica sul protocollo di consenso CryptoNote, 
proseguendo con una descrizione delle funzioni hash resistenti ad ASIC, con particolare attenzione a CryptoNight. 
Successivamente, si esaminerà RandomX, il nuovo algoritmo adottato da Monero, progettato per superare le limitazioni di CryptoNight.
In conclusione, invece, verranno analizzati gli attacchi e i problemi di privacy riscontrati in CryptoNight, 
che rappresentano una minaccia anche per RandomX.


\section{Tecnologia CryptoNote}\label{aspetti-tecnici-di-cryptonote}
Quanto segue prende spunto dal whitepaper ufficiale di CryptoNote \cite{cryptonotev2}.


\subsection{Privacy e Anonimato nel Cash Elettronico}\label{privacy-e-anonimato-nel-cash-elettronico}
Privacy e anonimato sono aspetti fondamentali nei pagamenti elettronici. I
pagamenti peer-to-peer mirano a rimanere nascosti agli occhi di terze
parti, una netta differenza rispetto alle banche tradizionali. In
generale le aziende non vogliono rivelare le loro transizioni interne e
le persone comuni desiderano mantenere riservate le proprie spese
personali.

\subsection{Proprietà di Irretracciabilità e Non Collegabilità}\label{proprieta-di-irretracciabilita-e-non-collegabilita}
T. Okamoto e K. Ohta hanno descritto sei criteri per un sistema di
denaro elettronico ideale, uno dei quali riguarda la privacy: \textit{la
relazione tra l'utente e i suoi acquisti deve essere irrintracciabile}
\cite{okamoto1991universal}. Dunque, per definire il concetto di sistema di pagamento
anonimo servono due proprietà:

\begin{itemize}
\item
  \textbf{Irretracciabilità}: per ogni transizione eseguita tutti i possibili
  mittenti devono avere la stessa probabilità di essere identificati.
\item
  \textbf{Non collegabilità}: per due qualsiasi transazioni in uscita deve essere
  impossibile dimostrare che siano state inviate dalla stessa persona
\end{itemize}


\subsubsection{Caso Bitcoin}\label{limiti-di-bitcoin}
Oltre ai problemi di inequità tra utenti causati dall'uso di ASIC, Bitcoin non soddisfa nessuno dei due criteri.
Per quanto riguarda l'irretracciabilità, le transazioni che avvengono, sono pubbliche e possono essere ricondotte a
un'unica origine e ad un unico destinatario.
Anche se due partecipanti effettuano transazioni in modo indiretto, un
metodo di ricerca del percorso ben progettato (ad esempio, l'algoritmo
\textit{A star} \cite{hart1968formal}) può rivelare l'origine e il destinatario
finale.

Per la non collegabilità, invece, da un attenta analisi della blockchain e da alcune ricerche
\cite{reid2013analysis}, \cite{analysis_bitcoin}, \cite{ron2013quantitative}, si potrebbe rilevare una
connessione tra gli utenti e le loro transazioni.

L'incapacità di Bitcoin di soddisfare queste due proprietà porta a
concludere che esso non rappresenta un sistema anonimo, ma piuttosto
pseudo-anonimo. 
A tal proposito, sono state proposte diverse soluzioni \cite{mixing_services},
\cite{secure_multiparty} basate sull'idea di mescolare diverse transazioni pubbliche
e inviarle tramite un indirizzo intermediario ma questo porterebbe un
altro inconveniente, ovvero una terza parte fidata.


\subsection{Protocolli di Firma e schemi di CryptoNote}\label{protocolli-di-firma-e-schemi-di-cryptonote}
Seguono ora degli schemi di transazioni completamente anonime che
soddisfano le condizioni di non irretracciabilità e non collegabilità.
Una caratteristica importante è l'autonomia: il mittente non è tenuto a
collaborare con altri utenti o terze parti per le transazioni.

Lo schema di CryptoNote si basa su una primitiva crittografica chiamata
\emph{group signature}, inventata da \emph{D. Chaun} e \emph{E. van
Heyst} \cite{chaum_van_heyst} che consente di firmare un messaggio per conto di un
gruppo.
L'idea è che dopo aver firmato, l'utente fornisce (per verificare) non la propria
chiave pubblica, ma le chiavi di tutti gli utenti del suo gruppo. Chi
verifica vede che il vero firmatario è un membro di questo gruppo, ma
non conosce la sua esatta identità.

Il protocollo originale prevedeva una Terza Parte Fiduciosa (Gestore del
Gruppo), il quale era l'unico che poteva risalire al reale firmatario. La
versione successiva, \emph{ring} \emph{signature}, introdotta da Rivest
\cite{rivest_et_al} , prevedeva uno schema ad anello autonomo senza
responsabile del gruppo e con revoca dell'anonimato.

Successivamente, sono apparse diverse modifiche, quella che viene
adottata su CryptoNote per larga parte si basa sullo studio
\emph{Traceable ring signature} di \emph{E. Fujisaki and K. Suzuki}
\cite{fujisaki_suzuki}.
Per distinguere l'algoritmo originale da quello
modificato nella versione su CryptoNote, quest'ultima
firma verrà chiamata \emph{one-time ring signature}, sottolineando la
capacità dell'utente di produrre una sola firma valida
con la chiave privata.

La proprietà di tracciabilità è stata indebolita, mantenendo però quella
di collegabilità (linkability) per fornire unicità: la chiave pubblica
potrebbe comparire in molti set di verifica esterni e la chiave privata
può essere utilizzata per generare una firma anonima unica. Nel caso di
un tentativo di doppia spesa, queste due firme saranno collegate, ma non
è necessario rivelare l'identità del firmatario.

\subsubsection{Introduzione alla One-Time Ring Signature}
Alla base dell'algoritmo di firma si usa EdDSA, sviluppato e
implementato da \emph{D.J. Bernstein} \cite{bernstein_et_al}, parametri comuni di
dominio sono: 
\begin{itemize}
  \item[-] q: numero primo; 
  \item[-] d: elemento of Fq; 
  \item[-] E: equazione della curva ellittica;
  \item[-] G: punto base; 
  \item[-] l: ordine primo del punto base; 
  \item[-] Hs: funzione hash crittografica \{0, 1\} * → Fq; 
  \item[-] Hp: funzione hash deterministica E(Fq) → E(Fq).
\end{itemize}

Al fine di ottenere una maggiore privacy, sono necessari alcuni nuovi
termini che non dovrebbero essere confusi con le entità di Bitcoin:

\begin{itemize}
  \item
    \textbf{private ec-key} è una chiave segreta standard di curva
    ellittica: un numero $a \in [1,l-1]$
  \item
    \textbf{public ec-key} è una chiave pubblica standard di curva
    ellittica: un punto $A=aG$;
  \item
    \textbf{one-time keypair} è una coppia di chiavi ec-private e
    ec-public;
  \item
    \textbf{private user key} è una coppia \emph{$(a, b)$} di due diverse
    chiavi ec-private;
  \item
    \textbf{tracking key} è una coppia \emph{$(a, B)$} di chiave ec-private
    e chiave ec-public \emph{(dove $B=bG$ e $a \neq b$)};
  \item
    \textbf{public user key} è una coppia \emph{$(A, B)$} di due chiavi
    ec-public derivate da \emph{$(a, b)$};
  \item
    \textbf{standard} \textbf{address} è una rappresentazione di una
    chiave utente pubblica mediante una stringa digitabile
    dall'utente con correzione degli errori.
  \end{itemize}

Contrariamente al modello di Bitcoin, in cui un utente possiede sia le
chiavi uniche private che pubbliche, nel modello proposto un mittente
genera una chiave one-time publica basata sull'indirizzo del
destinatario e su alcuni dati. In questo senso, una transazione in
entrata per lo stesso destinatario viene inviata a una chiave pubblica
monouso (non direttamente a un indirizzo univoco) e solo il destinatario
può recuperare la parte privata corrispondente per riscattare i suoi
fondi (utilizzando la sua chiave privata unica). Il destinatario può
spendere usando una firma ad anello, mantenendo anonima la sua proprietà
e la sua effettiva spesa.


\subsection{Transazioni in CryptoNote} \label{funzionamento-delle-transazioni}
Gli indirizzi Bitcoin classici, una volta pubblicati, diventano
identificatori inequivocabili per ogni pagamento in entrata,
collegandoli tra loro e associandoli al destinatario.

\begin{figure}[h]
  \centering
  \includegraphics[width=0.5\textwidth]{image6.png}
  \caption{Modello chiave/transazione tradizionale di Bitcoin.}
\end{figure}

Come anticipato precedentemente, in CryptoNote viene proposta una soluzione che consente all'utente di
pubblicare un singolo indirizzo e ricevere pagamenti non collegabili su esso. 
La destinazione di ciascun output (di default) è una chiave
pubblica unica, derivata dall'indirizzo del destinatario
e dall'inserimento di dati casuali da parte del
mittente.

\begin{figure}[h]
  \centering
  \includegraphics[width=0.5\textwidth]{image7.png}
  \caption{Modello di transazione CryptoNote.}
\end{figure}

Innanzitutto, il mittente esegue il protocollo di scambio Diffie-Hellman
per ottenere un segreto condiviso dai suoi dati e da una metà
dell'indirizzo. Successivamente calcola una chiave di
destinazione monouso, utilizzando questi segreti e la seconda metà. Per
questi due passaggi sono necessarie due chiavi ec-keys del destinatario;
quindi, un indirizzo CryptoNote standard è grande quasi il doppio di un
indirizzo Bitcoin. Il destinatario esegue anche il protocollo
Diffie-Hellman e poi recupera la chiave segreta corrispondente.

Una transazione standard procede come segue:

\begin{enumerate}
\def\labelenumi{\arabic{enumi}.}
\item
  Alice vuole inviare un pagamento a Bob, che ha pubblicato il suo
  indirizzo. Lo decomprime e ottiene la chiave utente pubblica di Bob
  \emph{$(A, B)$}.
\item
  Alice genera un numero casuale $r \in [1,l-1]$ e calcola la chiave
  pubblica one-time $P=H_s(rA)G+B$.
\item
  Alice usa $P$ come chiave di destinazione per l'output e
  inserisce anche il valore $R=rG$ (come parte del protocollo
  Diffie-Hellman) da qualche parte nella transazione. Alice può creare
  altri output con chiavi pubbliche uniche: chiavi diverse dei
  destinatari \emph{$(A_i,B_i)$} implicano $P_i$ diversi anche con
  lo stesso $r$.

  \begin{figure}[h]
    \centering
    \includegraphics[width=0.5\textwidth]{image3.png}
    \caption{Standard transaction structure.}
  \end{figure}
\item
  Bob controlla ogni transazione in arrivo con la sua chiave privata $(a,
  b)$, calcolando $P'=H_s(aR)G+B$. Se la transazione di Alice è presente,
  allora $aR=arG=rA$ e $P'=P$.
\item
  Ora Bob può recuperare la chiave privata una tantum corrispondente:
  $x=H_s(aR)+b$, così come $P=xG$. Può spendere questo output in qualsiasi
  momento firmando la transazione con $x$.

  \begin{figure}[h]
    \centering
    \includegraphics[width=0.5\textwidth]{image4.png}
    \caption{Check della transazione in ingresso.}
  \end{figure}
\end{enumerate}

Di conseguenza, Bob riceve pagamenti in entrata associati a chiavi
pubbliche monouso che non possono essere collegate per un osservatore
esterno.

\subsubsection{Firme ad anello}\label{firme-ad-anello}
Un protocollo basato su firme ad anello monouso consente agli utenti di
ottenere un'anonimato incondizionato. Purtroppo, i tipi
ordinari di firme crittografiche permettono di tracciare le transazioni
ai rispettivi mittenti e destinatari. La \emph{one-time ring signature}
usa diverse tipi di firme, consiste di quattro algoritmi (\textbf{GEN,
SIG, VER, LNK}).

\begin{itemize}
  \item
    \textbf{GEN} prende parametri pubblici e restituisce una coppia ec
    \emph{$(P, x)$} e una chiave pubblica $I$.
  \item
    \textbf{SIG} riceve un messaggio $m$, un insieme $S'$ di chiavi pubbliche
    $\{P_i\}_{i \neq s}$, le coppie \emph{$(P_s, x_s)$} e restituisce una firma $\sigma$ e un
    insieme $S = S' \cup \{P_s\}$.
  \item
    \textbf{VER} riceve un messaggio $m$, un insieme $S$, una firma $\sigma$ e
    restituisce "true" o "false".
  \item
    \textbf{LNK} riceve un insieme $I = \{I_i\}$, una firma $\sigma$ e restituisce
    "linked" o "indep".
\end{itemize}

Lo scopo principale del protocollo è il seguente: un utente produce una
firma che può essere verificata non da una singola chiave pubblica, ma
da un insieme di chiavi. Il vero firmatario è indistinguibile dagli
altri proprietari di chiavi fino a quando non produce la seconda firma
sotto la stessa coppia di chiavi.

\begin{figure}[h]
  \centering
  \includegraphics[width = 0.55\textwidth]{image5.png}
  \caption{Anonimità firme ad anello.}
\end{figure}

\begin{itemize}
  \item
    \textbf{GEN}: Il firmatario sceglie casualmente una chiave segreta $x \in
    [1,l-1]$ e calcola la chiave pubblica corrispondente $P=xG$. Inoltre,
    calcola un'altra chiave pubblica $I=xHp(P)$ chiamata
    "immagine della chiave".
  \item
    \textbf{SIG}: Il firmatario genera una firma ad anello one-time con
    una prova a conoscenza zero non interattiva. Seleziona un sottoinsieme
    casuale $S'$ di $n - 1$ chiavi pubbliche di altri utenti $P_i$,
    la propria coppia di chiavi $(x, P)$ e l'immagine della
    chiave $I$. Sia $1 \leq s \leq n$ l'indice segreto del firmatario in
    $S$ (in modo che la sua chiave sia $P_s$).\\
    Si sceglie casualmente un elemento casuale da $\{q_i | i = 1
    ... n\}$ e $\{w_i | i = 1 ... n, i \neq s\}$ da $(1 ... l)$ e effettua
    i seguenti passaggi:
    \[ L_i = \begin{cases} q_iG, & \text{se } i = s \\ q_iG+w_iP_i, & \text{se } i \neq s \end{cases} \]


\[ R_i = \begin{cases} q_iH_p(P_i), & \text{se } i = s \\ q_iH_p(P_{i})+ w_iI, & \text{se } i \neq s \end{cases} \]

Il prossimo passo è ottenere la sfida non interattiva:

\[ c = H_s(m, L_1, \ldots, L_n, R_1, \ldots, R_n) \]

Infine il firmatario calcola la risposta:

\[ c_i = \begin{cases} w_i, & \text{se } i \neq s \\ \left(c -\sum_{i=1}^{n} c_i\right) \mod l, & \text{se } i = s \end{cases} \]

\[ r_i = \begin{cases} q_i, & \text{se } i \neq s \\ q_s - c_sx \mod l ,& \text{se } i=s \end{cases} \]

La firma risultante è \[ \sigma= (I,c_1,\ldots,c_n,r_1,\ldots,r_n). \]

\item 
  \textbf{VER}: Chi sta verificando controlla la firma, ricostruendo:

\[ \begin{cases}
L_i' = r_iG + c_iP_i \\
R_i' = r_iH_p(P_i) + c_iI
\end{cases}
 \] Chi verifica controlla se
\[\sum_{i=1}^{n} ci =^? H_s(m,L_1',...,L_n',R_1',...R_n') \mod l \]

Se questa uguaglianza è vera, chi verifica esegue
l'algoritmo \textbf{LNK}, altrimenti respinge la firma.
\item 
  \textbf{LNK}: Chi verifica controlla se I è stata utilizzata in firme
passate (questi valori sono memorizzati nell'insieme I).
Un doppio utilizzo significa che sono state prodotte due firme con la
stessa chiave segreta.
\end{itemize}

Meccanismo del protocollo: utilizzando L-commitments, il firmatario
dimostra di conoscere un certo $x$ tale che almeno una $Pi = xG$.
Per rendere questa prova non ripetibile introduciamo l'immagine
della chiave come $I=xHp(P)$. Il firmatario utilizza gli stessi
coefficienti $(r_i,c_i)$ per dimostrare quasi la stessa cosa: egli conosce
un certo $x$ tale che almeno uno $Hp(P_i)=I \cdot x^{-1}$. Se $x \rightarrow I$ è iniettiva:

\begin{itemize}
\item 
  Nessuno può recuperare la chiave pubblica dall\'immagine della chiave e identificare il firmatario.
\item 
  Il firmatario non può fare due firme con I diverse e lo stesso x.
\end{itemize}

Con una firma ad anello one-time, Bob può efficacemente nascondere
l'output di Alice (cioè, il suo input) tra gli altri: tutti i possibili
spenditori saranno equiprobabili, anche se Alice non ha più informazioni
di qualsiasi osservatore. Bob specifica n-1 outputs, non sapendo se
alcuni di questi sono stati spesi.
Un output può essere quindi utilizzato in migliaia di firme come fattore di
ambiguità e mai come obiettivo di occultamento. Il controllo di doppia
spesa avviene nella fase LNK quando si cerca
nell'insieme delle immagini di chiave utilizzate.

Bob può scegliere il grado di ambiguità autonomamente: n = 2 significa
che avrà speso l'output con una probabilità del 50\%, n
= 100 dà il 1\%. La dimensione della firma risultante è lineare O(n),
quindi l'anonimato costa a Bob una dimensione di
transazione più grande e commissioni più alte.

Combinando entrambi i metodi (chiavi di transazione one-time e firme ad
anello one-time), Bob raggiunge un nuovo livello di privacy rispetto
allo schema originale di Bitcoin. Gli basta memorizzare una sola chiave
privata (a, b) e generare una chiave pubblica (A, B) per iniziare a
ricevere e inviare transazioni anonime. Per ogni output Bob recupera
coppie di chiavi di transazione uniche (pi, Pi) che non possono essere
collegate tra loro o alla sua chiave pubblica. Può spendere ognuna di
esse, firmando ogni input con una firma ad anello non tracciabile.


\subsection{Miglioramenti nella PoW rispetto a Bitcoin}\label{miglioramenti-nella-pow-rispetto-a-bitcoin}
Oltre a migliorare l'aspetto della privacy, Cryptonote si prefigge anche l'obiettivo di 
ridurre l'impiego di FPGA ed ASIC che centralizzano il mining.

Il protocollo PoW originale di Bitcoin utilizza la funzione SHA-256, funzione che
consiste in operazioni logiche di base, e si fonda sulla velocità computazionale del processore.
Tuttavia, si noti che i computer moderni non sono limitati solo
dal numero di operazioni al secondo, ma anche dalla dimensione della
memoria. 

Mentre alcuni processori possono essere notevolmente più veloci
di altri \cite{mining_hardware} o possono utilizzare implementazioni multicore, 
le dimensioni della memoria sono generalmente meno variabili tra le macchine.

L'idea principale di CryptoNote è costruire un algoritmo che alloca un
ampio blocco di dati all'interno della
memoria e "accedere a una sequenza imprevedibile di posizioni" in esso.
Il blocco dovrebbe essere sufficientemente grande per rendere più
vantaggioso conservare i dati piuttosto che ricalcolarli ad ogni
accesso. L'algoritmo dovrebbe inoltre impedire il
parallelismo interno, quindi $N$ thread simultanei dovrebbero richiedere $N$
volte più memoria contemporaneamente.

Questo è esattamente ciò che è stato ottenuto grazie all'algoritmo di PoW 
\textbf{CryptoNight}, che verrà analizzato nel prossimo capitolo.

\subsection{Equità nella distribuzione} \label{equituxe0-nella-distibuzione}
\subsubsection{Limite superiore e formula delle ricompense dei blocchi}
Il limite superiore per l'ammontare complessivo delle
monete digitali CryptoNote è anche digitale:
\[\text{MSupply} = 2^{64} - 1\] unità atomiche. Questa è una restrizione
naturale basata solo su limiti di implementazione, non su intuizioni
come "N monete dovrebbero essere sufficienti per chiunque".

Per garantire la regolarità del processo di emissione, viene utilizzata
la seguente formula per le ricompense dei blocchi: \[
\text{BaseReward} = (\text{MSupply} - A) >> 18 \]

dove A è l'ammontare di monete generate precedentemente

\subsubsection{Algoritmo di adattamento della difficoltà}
CryptoNote contiene inoltre un algoritmo di targeting che cambia la difficoltà
di ogni blocco. Questo migliora il tempo di reazione del sistema quando
la potenza di calcolo della rete cresce o diminuisce intensamente,
preservando un tasso di blocco costante. Il metodo originale di Bitcoin
calcola il rapporto tra la difficoltà effettiva e quella target tra gli
ultimi 2016 blocchi e lo utilizza come moltiplicatore per la difficoltà
attuale. Ovviamente questo è inadatto per ricalcoli rapidi (a causa
dell'inerzia elevata) e porta a oscillazioni.

L'idea generale dietro l'algoritmo è sommare tutto il
lavoro completato dai nodi e dividerlo per il tempo impiegato per
completare il lavoro. La misura del lavoro sono i valori di difficoltà
corrispondenti in ogni blocco.

\subsubsection{Dimensione delle transazioni}
Gli utenti pagano gli altri per memorizzare la blockchain e dovrebbero
avere il diritto di votare per la sua dimensione. Ogni miner si
confronta con il compromesso tra bilanciare i costi e il profitto dalle
commissioni, quindi stabilisce il proprio "limite flessibile" per la
creazione dei blocchi. Inoltre, la regola fondamentale per la dimensione
massima del blocco è necessaria per evitare che la blockchain venga
inondatata da transazioni fasulle, tuttavia questo valore non dovrebbe
essere codificato duramente. Sia MN il valore mediano delle dimensioni
degli ultimi N blocchi.

Allora il "limite rigido" per la dimensione dei blocchi accettati è 2 ·
MN.

Un miner ha ancora la possibilità di riempire un blocco con le sue
transazioni senza commissioni fino alla dimensione massima di 2 MB.
Anche se solo la maggioranza dei miners può spostare il valore mediano,
esiste comunque la possibilità di gonfiare la blockchain e produrre un
carico aggiuntivo sui nodi. Per scoraggiare i partecipanti malevoli dal
creare blocchi grandi, introduciamo una funzione di penalità:

\[
\text{NewReward} = \text{BaseReward} \times \left( \frac{\text{DimBlocco}}{MN} - 1 \right)^2
\]

Questa regola viene applicata solo quando la DimBlocco è maggiore della
dimensione minima del blocco gratuito che dovrebbe essere vicina a \[
\max(10\, \text{kb}, M_N \cdot 110\%) \] I miners sono autorizzati a
creare blocchi di ``dimensioni usuali'' e persino a superarle con
profitto quando le commissioni complessive superano la penalità.

Tuttavia, è improbabile che le commissioni crescano in modo quadratico a
differenza del valore della penalità, quindi ci sarà un equilibrio.