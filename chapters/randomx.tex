\chapter{RandomX}

Il seguente capitolo entra nel dettaglio dell'implementazione del successore di CryptoNight, denominato RandomX. CryptoNight è diventato obsoleto poiché non è più resistente agli ASIC. Di conseguenza, RandomX è stato sviluppato esternamente a CryptoNote, ma poiché è attualmente l'ultima versione degli algoritmi PoW ASIC-resistant, abbiamo deciso di includerlo. Tutte le informazioni sono prese dalla pagina GitHub ufficiale \cite{randomx}.

\section{Design di Randomx}
Per minimizzare il vantaggio di hardware specializzati come gli ASIC, già precedentemente discussi, un algoritmo di \textbf{Proof of Work} (\textbf{PoW}) deve potersi legare ai dispositivi esistenti il cui uso risulti essere ampiamente diffuso. Non a caso, si focalizza sull'utilizzo delle \textbf{CPU} per i seguenti motivi:

\begin{itemize}
    \item \textbf{Accessibilità}: Le CPU, essendo meno specializzate, sono più diffuse e accessibili. Un algoritmo basato su CPU è più \textbf{egalitario} e permette a più partecipanti di unirsi alla rete, contrariamente a ciò che accadeva con gli ASIC in quanto il loro notevole costo permetteva una centralizzazione del mining nelle mani di pochi;
    \item \textbf{Istruzioni Hardware Comuni}: CPU diverse \textbf{condividono} un grande \textit{subset} di istruzioni hardware native;
    \item \textbf{Documentazione e Compilatori}: Tutti i principali set di istruzioni CPU sono ben \textbf{documentati} con diversi compilatori \textbf{open-source} disponibili.
\end{itemize}

\subsection{Prova di Lavoro Dinamica} 
L'idea alla base di una \textbf{Proof of Work} basata su CPU è l'impiego di un \textit{lavoro dinamico} sfruttando il fatto che queste accettano come input non solo \textbf{dati}, come le tipiche funzioni di hash crittografiche, ma anche il \textbf{codice}. Ciò evita che le sequenza delle operazioni sia fissa e quindi più facilmente eseguibile da un circuito integrato specializzato. Una \textbf{Proof of Work dinamica} consiste in 4 steps:
\begin{enumerate}
    \item Generazione di un programma casuale;
    \item Traduzione nel codice macchina nativo della CPU;
    \item Esecuzione del programma;
    \item Trasformazione dell'output del programma in un valore crittograficamente sicuro.
\end{enumerate}

\subsubsection{Generazione di un programma casuale}
Inizialmente la progettazione di una Proof of Work si basava sulla generazione di un programma in linguaggi ad alto livello, come C o Javascript, ma questi per via della loro sintassi complessa, implicavano notevoli costi in termini di tempo.

Il modo più veloce per generare un programma casuale è utilizzare un generatore senza logica, riempiendo semplicemente un \textbf{buffer con dati casuali}. Questo richiede la progettazione di un linguaggio di programmazione senza sintassi, in cui tutte le stringhe di bit casuali rappresentano programmi validi.

\subsubsection{Traduzione del Programma in Codice Macchina}
Per generare il codice macchina il più velocemente possibile, il nostro set di istruzioni deve essere il più vicino possibile all'hardware nativo, ma allo stesso tempo deve risultare abbastanza generico in modo da non limitare l'algoritmo a una specifica architettura CPU.

\subsubsection{Esecuzione del Programma}
L'esecuzione del programma dovrebbe utilizzare il \textbf{maggior numero} possibile di \textbf{componenti della CPU}. Alcune delle caratteristiche che dovrebbero essere sfruttate sono:
\begin{itemize}
    \item Cache multi-livello (L1, L2, L3);
    \item Cache  \textmu op;
    \item Unità logica aritmetica (ALU);
    \item Unità a virgola mobile (FPU);
    \item Controller di memoria;
    \item Parallelismo a livello di istruzione;
    \item Esecuzione fuori ordine;
    \item Esecuzione speculativa;
    \item Rinominazione dei registri.
\end{itemize}

\subsubsection{Calcolo del Risultato Finale}
\textbf{Blake2b} è una funzione di hashing crittograficamente sicura, progettata per essere veloce nel software, soprattutto sui moderni processori a 64 bit, dove è circa tre volte più veloce di SHA-3. Proprio per questo è ideale per essere utilizzata in una \textbf{Proof of Work} basata su \textbf{CPU}.

Per quanto riguarda, invece, l'elaborazione di grandi quantità di dati in modo crittograficamente sicuro, l'\textbf{Advanced Encryption Standard }(\textbf{AES}) può fornire la massima velocità di elaborazione perché molte CPU moderne supportano l'accelerazione hardware di queste operazioni.

\subsection{"Easy program problem"}
Il problema dell'\textit{easy program} si basa sul fatto che quando viene generato un programma casuale, si potrebbe decidere di eseguirlo solo in caso sia favorevole. Questa strategia è fattibile per due motivi principali:

\begin{itemize}
    \item \textbf{Distribuzione del Tempo di Esecuzione}: I tempi di esecuzione dei programmi generati casualmente seguono tipicamente una \textit{distribuzione log-normale}. Questi possono rapidamente analizzati e in caso di tempo di esecuzione superiore alla media, l'esecuzione può essere saltata e può essere generato un nuovo programma. Se quest'ultima operazione risulta essere economica, può migliorare significativamente le prestazioni.
    \item \textbf{Ottimizzazione delle Caratteristiche}: Un'implementazione potrebbe scegliere di ottimizzare un sottoinsieme delle caratteristiche necessarie per l'esecuzione del programma. Ad esempio, si può decidere di eliminare il supporto per alcune operazioni (come la divisione) o di implementare alcune sequenze di istruzioni in modo più efficiente. In seguito, i programmi generati verrebbero analizzati ed eseguiti solo se soddisfano i requisiti specifici dell'implementazione ottimizzata.
\end{itemize}

Queste strategie di ricerca di programmi con particolari proprietà vanno in \textbf{contrasto} con gli obiettivi di questa \textbf{Proof of Work}, quindi devono essere eliminate:
\begin{itemize}
    \item \textbf{Soluzione}: Esecuzione di una sequenza di \textbf{N programmi casuali}, in modo che ogni programma sia generato dall'\textbf{output del precedente}. L'output del programma finale viene quindi utilizzato come risultato.
    \item \textbf{Principio}: Una volta eseguito il primo programma, un miner deve decidere \textbf{se impegnarsi a terminare l'intera catena} (che può includere programmi sfavorevoli) o ricominciare da capo e sprecare lo sforzo impiegato sulla catena non completata;
    \item \textbf{Vantaggio}: Uniformare il tempo di esecuzione per l'intera catena, poiché la deviazione relativa di una somma di tempi di esecuzione distribuiti identicamente è ridotta.
\end{itemize}

\subsection{Tempo di verifica}
Poiché lo scopo del \textbf{Proof of Work} è di essere utilizzato in una \textbf{rete peer-to-peer} senza fiducia, i partecipanti alla rete devono essere in grado di verificare rapidamente se una prova è valida o meno. Ciò pone un \textbf{limite} superiore alla \textbf{complessità dell'algoritmo} di \textbf{Proof of Work}. In particolare, abbiamo fissato l'obiettivo per RandomX di essere almeno altrettanto veloce da verificare quanto la funzione di hash CryptoNight, che mira a sostituire.

\subsection{Memory-hardness}
Oltre alle risorse computazionali pure, come \textbf{ALU} e \textbf{FPU}, le \textbf{CPU} di solito hanno accesso a una grande quantità di memoria sotto forma di DRAM. Le prestazioni del sottosistema di memoria sono tipicamente ottimizzate per adattarsi alle capacità di calcolo.

Per utilizzare la \textbf{memoria esterna} così come i controller di memoria on-chip, l'\textbf{algoritmo di Proof of Work} dovrebbe accedere a un grande \textbf{buffer} di memoria (chiamato "\textbf{Dataset}"). Il Dataset deve essere:

\begin{itemize}
    \item \textbf{più grande} di quanto possa essere memorizzato on-chip (per richiedere memoria esterna);
    \item \textbf{dinamico} (per richiedere memoria scrivibile)
\end{itemize}

Idealmente, la dimensione del Dataset dovrebbe essere di almeno 4 GiB. Tuttavia, a causa dei vincoli sul tempo di verifica la dimensione utilizzata da RandomX è stata selezionata a 2080 MiB.

\section{Architettura della macchina virtuale}

Questa sezione descrive la progettazione della macchina virtuale (VM) di RandomX.

\subsection{Set di istruzioni}

RandomX utilizza una codifica delle istruzioni a lunghezza fissa con 8 byte per istruzione. Questo permette di includere un valore immediato a 32 bit nella parola dell'istruzione.

\subsection{Programma}

Il programma eseguito dalla VM ha la forma di un loop composto da 256 istruzioni casuali.

\begin{itemize}
  \item 256 istruzioni sono sufficientemente lunghe da fornire un gran numero di programmi possibili e abbastanza spazio per i branch. Il numero di programmi diversi che possono essere generati è limitato a $2^{512} = 1.3 \times 10^{154}$, che è il numero di valori di seme possibili del generatore casuale.
  \item 256 istruzioni sono abbastanza brevi affinché le CPU ad alte prestazioni possano eseguire una iterazione in un tempo simile a quello necessario per recuperare i dati dalla DRAM. Questo è vantaggioso perché permette di sincronizzare e prefetchare completamente gli accessi al Dataset (vedi capitolo 2.9).
  \item Poiché il programma è un loop, può sfruttare la $\mu$op cache che è presente in alcune CPU x86. Eseguire un loop dalla $\mu$op cache permette alla CPU di spegnere i decoder delle istruzioni x86, il che dovrebbe aiutare a equilibrare l'efficienza energetica tra x86 e architetture con decodifica delle istruzioni semplice.
\end{itemize}

\subsection{Registri}

La VM utilizza 8 registri interi e 12 registri floating-point. Questo è il massimo che può essere allocato come registri fisici in x86-64, che ha il minor numero di registri architetturali tra le comuni architetture CPU a 64 bit. Utilizzare più registri metterebbe le CPU x86 in svantaggio poiché dovrebbero utilizzare la memoria per memorizzare i contenuti dei registri della VM.

\subsection{Operazioni intere}

RandomX utilizza tutte le operazioni intere primitive che hanno un'elevata entropia di output: addizione (IADD\_RS, IADD\_M), sottrazione (ISUB\_R, ISUB\_M, INEG\_R), moltiplicazione (IMUL\_R, IMUL\_M, IMULH\_R, IMULH\_M, ISMULH\_R, ISMULH\_M, IMUL\_RCP), exclusive or (IXOR\_R, IXOR\_M) e rotazione (IROR\_R, IROL\_R).

\subsection{Operazioni floating-point}

RandomX utilizza operazioni floating-point (virgola mobile) a doppia precisione, che sono supportate dalla maggior parte delle CPU e richiedono hardware più complesso rispetto alla singola precisione. Tutte le operazioni vengono eseguite come operazioni vettoriali a 128 bit, anch'esse supportate da tutte le principali architetture CPU.

RandomX utilizza cinque operazioni garantite dallo standard IEEE 754 per dare risultati arrotondati correttamente: addizione, sottrazione, moltiplicazione, divisione e radice quadrata. Tutte le 4 modalità di arrotondamento definite dallo standard vengono utilizzate.

\subsection{Branch}

Le CPU moderne investono molta area del die e energia per gestire i branch. Questo include:

\begin{itemize}
  \item Branch predictor unit
  \item Stati di checkpoint/rollback che permettono alla CPU di recuperare in caso di una predizione errata del branch.
\end{itemize}

Per sfruttare i progetti speculativi, i programmi casuali dovrebbero contenere branch. Tuttavia, se la predizione dei branch fallisce, le istruzioni eseguite speculativamente vengono scartate, il che comporta una certa quantità di energia sprecata con ogni predizione errata. Pertanto dovremmo mirare a minimizzare il numero di predizioni errate.

Inoltre, i branch nel codice sono essenziali perché riducono significativamente il numero di ottimizzazioni statiche che possono essere effettuate. Ad esempio, considera la seguente sequenza di istruzioni x86:
\begin{verbatim}
    ...
branch_target_00:
    ...
    xor r8, r9
    test r10, 2088960
    je branch_target_00
    xor r8, r9
    ...
\end{verbatim}
Le operazioni XOR normalmente si annullerebbero, ma non possono essere ottimizzate via a causa del branch perché il risultato sarà diverso se il branch viene preso. Analogamente, l'istruzione ISWAP\_R potrebbe essere sempre ottimizzata staticamente se non fosse per i branch.

In generale, i branch casuali devono essere progettati in modo tale che:

1. Non siano possibili loop infiniti.
2. Il numero di branch predetti erroneamente sia piccolo.
3. La condizione del branch dipenda da un valore a runtime per disabilitare le ottimizzazioni statiche del branch.

\subsection{Parallelismo a livello di istruzione}

Le CPU migliorano le loro prestazioni utilizzando diverse tecniche che sfruttano il parallelismo a livello di istruzione del codice eseguito. Queste tecniche includono:

\begin{itemize}
  \item Avere più unità di esecuzione che possono eseguire operazioni in parallelo (\textit{superscalar execution}).
  \item Eseguire istruzioni non nell'ordine del programma, ma nell'ordine della disponibilità degli operandi (\textit{out-of-order execution}).
  \item Prevedere quale direzione prenderanno i branch per migliorare i benefici sia dell'esecuzione superscalare che di quella out-of-order.
\end{itemize}

RandomX beneficia di tutte queste ottimizzazioni.

\subsection{Scratchpad}

Lo Scratchpad è una "zona di lavoro" utilizzata come memoria di lettura-scrittura. La sua dimensione è stata selezionata per adattarsi completamente nella cache della CPU.

\subsection{Dataset}

Poiché lo Scratchpad viene solitamente memorizzato nella cache della CPU, solo gli accessi al Dataset utilizzano i controller di memoria.

RandomX legge casualmente dal Dataset una volta per iterazione del programma (16384 volte per risultato hash). Poiché il Dataset deve essere memorizzato nella DRAM, fornisce un limite naturale alla parallelizzazione, poiché la DRAM non può effettuare più di circa 25 milioni di accessi casuali al secondo per gruppo di banchi. Ogni gruppo di banchi indirizzabili separatamente consente un throughput di circa 1500 H/s.

Tutti gli accessi al Dataset leggono una linea di cache della CPU (64 byte) e sono completamente prefetchati. Il tempo per eseguire una iterazione del programma descritta nel capitolo 4.6.2 della Specifica è circa lo stesso della latenza tipica di accesso alla DRAM (50-100 ns).

\section{Funzioni personalizzate}

\subsection{AesGenerator1R}

\texttt{AesGenerator1R} è stato progettato per la generazione più veloce possibile di dati pseudocasuali per riempire lo Scratchpad. Sfrutta l'AES accelerato via hardware nelle CPU moderne. Viene eseguito solo un round AES per ogni 16 byte di output, il che si traduce in un throughput superiore a 20 GB/s nella maggior parte delle CPU moderne.

AesGenerator1R fornisce una buona distribuzione dell'output a condizione che venga inizializzato con uno stato iniziale sufficientemente 'casuale' (vedi Appendice F).

\subsection{AesGenerator4R}

\texttt{AesGenerator4R} utilizza 4 round AES per generare dati pseudocasuali per l'inizializzazione del Program Buffer. Poiché 2 round AES sono sufficienti per l'effetto valanga completo di tutti i bit di input, AesGenerator4R ha eccellenti proprietà statistiche (vedi Appendice F) mantenendo comunque prestazioni molto buone.

La natura reversibile di questo generatore non è un problema poiché lo stato del generatore viene sempre inizializzato utilizzando l'output di una funzione di hash non reversibile (Blake2b).

\subsection{AesHash1R}

\texttt{AesHash} è stato progettato per il calcolo più veloce possibile dell'impronta digitale dello Scratchpad. Interpreta lo Scratchpad come un insieme di chiavi di round AES, quindi è equivalente alla crittografia AES con 32768 round. Vengono eseguiti due round extra alla fine per garantire l'effetto valanga di tutti i bit dello Scratchpad in ogni lane.

La natura reversibile di AesHash1R non è un problema per due motivi principali:

\begin{itemize}
  \item Non è possibile controllare direttamente l'input di AesHash1R.
  \item L'output di AesHash1R viene passato alla funzione di hash Blake2b, che non è reversibile.
\end{itemize}

\subsection{SuperscalarHash}

\texttt{SuperscalarHash} è stato progettato per consumare quanta più energia possibile mentre la CPU attende che i dati vengano caricati dalla DRAM. La latenza target di 170 cicli corrisponde alla latenza usuale della DRAM di 40-80 ns e alla frequenza di clock di 2-4 GHz. I dispositivi ASIC progettati per il mining in modalità leggera con memoria a bassa latenza saranno limitati da SuperscalarHash durante il calcolo degli elementi del Dataset e la loro efficienza sarà compromessa dall'alto consumo energetico di SuperscalarHash.

La funzione SuperscalarHash media contiene un totale di 450 istruzioni, di cui 155 sono moltiplicazioni a 64 bit. In media, la catena di dipendenze più lunga è composta da 95 istruzioni. Un progetto ASIC per il mining in modalità leggera, con 256 MiB di memoria on-die e una latenza di 1 ciclo per tutte le operazioni, avrà bisogno in media di 95 * 8 = 760 cicli per costruire un elemento del Dataset, assumendo una parallelizzazione illimitata. Dovrà eseguire 155 * 8 = 1240 moltiplicazioni a 64 bit per elemento, il che consumerà energia comparabile al caricamento di 64 byte dalla DRAM.



\section{Descrizione dell'algoritmo}

L'algoritmo RandomX accetta due valori di input:

\begin{itemize}
    \item Stringa \texttt{K} di dimensione 0-60 byte (chiave)
    \item Stringa \texttt{H} di lunghezza arbitraria (valore da hashare)
\end{itemize}

e produce un risultato di 256 bit \texttt{R}.

L'algoritmo consiste nei seguenti passaggi:

\begin{enumerate}
    \item Il Dataset viene inizializzato utilizzando il valore della chiave \texttt{K}.
    \item Il seed di 64 byte \texttt{S} viene calcolato come \texttt{S = Hash512(H)}.
    \item Sia \texttt{gen1 = AesGenerator1R(S)}.
    \item Lo Scratchpad viene riempito con \texttt{RANDOMX\_SCRATCHPAD\_L3} byte casuali utilizzando il generatore \texttt{gen1}.
    \item Sia \texttt{gen4 = AesGenerator4R(gen1.state)} (usare lo stato finale di \texttt{gen1}).
    \item Il valore del registro della VM \texttt{fprc} è impostato a 0 (modalità di arrotondamento predefinita).
    \item La VM viene programmata utilizzando \texttt{128 + 8 * RANDOMX\_PROGRAM\_SIZE} byte casuali utilizzando il generatore \texttt{gen4}.
    \item La VM viene eseguita.
    \item Un nuovo seed di 64 byte viene calcolato come \texttt{S = Hash512(RegisterFile)}.
    \item Imposta \texttt{gen4.state = S} (modifica lo stato del generatore).
    \item I passaggi 7-10 vengono eseguiti un totale di \texttt{RANDOMX\_PROGRAM\_COUNT} volte. L'ultima iterazione salta i passaggi 9 e 10.
    \item L'impronta digitale dello Scratchpad viene calcolata come \texttt{A = AesHash1R(Scratchpad)}.
    \item I byte 192-255 del Registro File vengono impostati al valore di \texttt{A}.
    \item Il risultato viene calcolato come \texttt{R = Hash256(RegisterFile)}.
\end{enumerate}

L'input della funzione \texttt{Hash512} nel passaggio 9 è il seguente:

\begin{verbatim}
 +---------------------------------+
 |         registri r0-r7          | (64 byte)
 +---------------------------------+
 |         registri f0-f3          | (64 byte)
 +---------------------------------+
 |         registri e0-e3          | (64 byte)
 +---------------------------------+
 |         registri a0-a3          | (64 byte)
 +---------------------------------+
\end{verbatim}

L'input della funzione \texttt{Hash256} nel passaggio 14 è il seguente:

\begin{verbatim}
 +---------------------------------+
 |         registri r0-r7          | (64 byte)
 +---------------------------------+
 |         registri f0-f3          | (64 byte)
 +---------------------------------+
 |         registri e0-e3          | (64 byte)
 +---------------------------------+
 |      AesHash1R(Scratchpad)      | (64 byte)
 +---------------------------------+
\end{verbatim}


% TODO
% aggiungere due ordini di profondità di design.md OK
% aggiungere  algorithm description di specs.md