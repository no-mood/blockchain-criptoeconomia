\chapter{Privacy}
\newtheorem{definition}{Definition}
Preservare la privacy degli utenti è fondamentale nelle reti informatiche e diventa ancor più cruciale nei sistemi pubblici, aperti e decentralizzati, come le blockchain. 

Tre sono i principali rischi per la privacy:
\begin{enumerate}
    \item \textbf{Locali}, dovuti all'uso di PC/Smartphone (browser history, cookies, blockchain client).
    \item \textbf{Network} (informazioni rivelate dall'indirizzo IP, man-in-the-middle, condivisione inavvertita dei dati privati)
    \item \textbf{Blockchain} (riutilizzo degli stessi address, analisi a ritroso delle transazioni).
\end{enumerate}

Quest'ultima può essere facilmente compromessa attraverso il clustering degli indirizzi. Questa tecnica consiste nel trovare e raggruppare gli indirizzi in wallet, ossia gruppi di indirizzi presumibilmente appartenenti a un singolo soggetto. Il clustering si realizza tramite l’analisi della blockchain e l’utilizzo di approcci euristici, descritti da Jonas David Nick nel suo lavoro “Data-Driven De-Anonymization in Bitcoin”.

\section{Euristiche}
Le euristiche sono misurazioni basate sull’analisi del protocollo e sull’esperienza. Non sono sempre accurate, ma se ben calibrate, possono fornire un'indicazione del livello di attendibilità dei risultati ottenuti. 

La prima euristica è chiamata “Multi-Input Heuristic”. In accordo con quanto affermato da Satoshi Nakamoto, inventore di Bitcoin, questa regola mostra come tutti gli indirizzi in input di una transazione provengano dallo stesso wallet, poiché l’autore della transazione possiede le chiavi private di tutti gli indirizzi in input. Tuttavia, l’autore non è necessariamente una singola persona; potrebbe trattarsi di un exchange, un mixer o un wallet online. 

La seconda euristica, denominata “Shadow Heuristic”, si basa su come i wallet gestiscono le transazioni con resto. E' bene osservare che nella maggior parte dei wallet, quando un utente invia una transazione, viene generata una nuova coppia di chiavi (pubblica e privata) per gestire l'output di resto. In una tipica transazione con un input e due output, uno degli output rappresenta l'indirizzo di destinazione e l'altro è l'indirizzo di resto. Supponendo che l'indirizzo di destinazione appartenga a un commerciante o a un'entità che riceve frequentemente pagamenti, questo indirizzo rimarrà costante in diverse transazioni. L'altro output, che cambia ad ogni transazione, è l'indirizzo di resto generato dal wallet. Da qui si evince che le chiavi pubbliche di invio e di resto siano nello stesso wallet. Di conseguenza, un potenziale indirizzo di resto può essere identificato come un indirizzo che appare in massimo due transazioni: una come output (quando riceve il resto) e una come input (quando il resto viene speso).

La terza euristica, la “Consumer Heuristic”, è un'altra tecnica utilizzata nell'analisi della blockchain per identificare le transazioni di resto e collegare indirizzi appartenenti allo stesso wallet, basandosi sul comportamento tipico degli utenti privati nel gestire le loro transazioni. Gli utenti privati tendono a effettuare transazioni con un solo destinatario alla volta. Pertanto, le transazioni effettuate da questi utenti hanno tipicamente al massimo due output: uno per il destinatario e uno per il resto. La sfida è identificare quale dei due output è il resto, analizzando le transazioni successive. Se in una transazione successiva, uno degli output dubbi invia denaro a più destinatari, è probabile che questo output sia il resto. Questo perché il resto, una volta tornato nel wallet dell'utente, può essere utilizzato in future transazioni che potrebbero avere più di un output. L'altro output, che non viene utilizzato in transazioni con più destinatari, è più probabilmente quello del destinatario originale della transazione iniziale.

L'ultima euristica è la “Optimal Change Heuristic”, basata sul fatto che i wallet cercano di ottimizzare la gestione del resto per ridurre le dimensioni delle transazioni e le conseguenti commissioni, minimizzando il numero di input e output necessari per ogni transazione. Da questo segue una regola generale in cui, come spiegato da Jonas Nick, se c'è un unico output con un valore inferiore a qualsiasi input, allora quello sarà il resto della transazione.

Questi approcci, insieme a principi più complessi, sono impiegati dai principali strumenti open source, come BitCluster e WalletExplorer, per supportare strategie efficaci di clustering. Combinando questi strumenti con un monitoraggio costante delle transazioni su piattaforme online per la visualizzazione della blockchain, o scaricando una copia della blockchain e installando un block explorer locale, è possibile ottenere ottimi risultati. Questi metodi permettono di collegare indirizzi ignoti a indirizzi noti, ricostruendo indirettamente l'identità dei proprietari o almeno i movimenti di denaro.

\section{Soluzioni}
Esistono diverse soluzioni per affrontare il problema della privacy nelle transazioni blockchain. Tra di esse ricordiamo la Ring Confidential Transaction, che permette a chi paga di dimostrare di avere a disposizione il denaro, ma senza indicare in chiaro l'importo della transazione; la Stealth Address, ovvero l'utilizzo di indirizzi indivisibili al fine di rendere difficile risalire ai wallet coinvolti nella transazione; Kovri, che consente agli utenti di nascondere il proprio indirizzo IP; lo sviluppo stesso di wallet e protocolli che implementano politiche di gestione del resto più complesse, al fine di rendere le transazioni meno suscettibili all'analisi del clustering. La soluzione su cui ci vogliamo soffermare riguarda l'utilizzo di tecniche avanzate di anonimizzazione, come le \textbf{firme ad anello}, i protocolli di \textbf{mixing} o \textbf{CoinJoin}, che offuscano le relazioni tra input e output delle transazioni, rendendo più difficile collegare gli indirizzi di un singolo utente.

Quando si parla di \textbf{mixer} o \textbf{tumbler}, si fa riferimento a servizi gestiti da terzi che “mescolano” le criptomonete inviate al fine di renderle anonime e difficili da tracciare. Questi servizi “ripuliscono” il denaro facendo in modo che passi attraverso vari indirizzi non collegati tra loro, se non tramite gli algoritmi interni del servizio stesso.
Il funzionamento dei servizi di mixing è piuttosto semplice: l'utente invia l'importo che desidera “ripulire” all'indirizzo Bitcoin fornito dal mixer. Il mixer, dopo aver trattenuto una piccola commissione, invia l'importo all'indirizzo indicato dall'utente, ma questi fondi provengono dai versamenti di altri utenti. In questo modo, diventa molto difficile, se non impossibile, collegare i due indirizzi tramite un'analisi della blockchain.
I migliori servizi di mixing offrono ulteriori funzionalità per aumentare l'anonimato, come il ritardo temporale e la scomposizione delle transazioni. Ad esempio, alcuni servizi, come BitMixer, permettono di dividere la transazione in più parti, ognuna delle quali viene inviata all'indirizzo di destinazione con un ritardo temporale stabilito dall'utente. Questo significa che può trascorrere anche qualche giorno tra la ricezione della prima parte dei fondi e il completamento della transazione. 
Naturalmente, il servizio ha un costo, sebbene limitato: sulla transazione viene applicata una piccola commissione, fissa o variabile, che viene ulteriormente anonimizzata attraverso vari metodi.
La sicurezza e l'anonimità di questi servizi centralizzati è ovviamente discutibile. Gli utenti non hanno alcuna garanzia che il mixer restituisca il loro denaro o che le monete rese non siano state segnate in qualche modo. Un altro aspetto da considerare quando si usa un mixer è che l'indirizzo IP e l'indirizzo Bitcoin potrebbero essere registrati da una terza parte. 

Un'altra possibile soluzione è l'utilizzo di operazioni di \textbf{CoinJoin}, che permettono di superare le criticità e i rischi associati ai mixer. CoinJoin consente di combinare la propria transazione con quella di altri utenti, senza dover riporre fiducia in un ente terzo, potenzialmente malevolo. Il principio di CoinJoin è che più utenti si coordinino per creare una singola transazione, ciascuno fornendo i propri input e output desiderati. Poiché tutti gli input vengono mescolati, diventa impossibile determinare con certezza quale output appartiene a quale utente. Al massimo, si può dire che un partecipante ha fornito uno degli input e potrebbe essere il proprietario di uno degli output, ma anche questo non è garantito.
Un coordinatore raccoglie tutte le informazioni necessarie, crea la transazione e la fa firmare a ciascun partecipante prima di trasmetterla alla rete. Una volta firmata, la transazione non può essere modificata senza diventare invalida, eliminando così il rischio che il coordinatore possa fuggire con i fondi.
I vantaggi di CoinJoin sono principalmente due: l'incremento della privacy e il costo contenuto. CoinJoin offre un elevato livello di privacy combinando le transazioni di più utenti, rendendo difficile tracciare l'origine e la destinazione dei fondi. Inoltre, non introduce costi aggiuntivi significativi rispetto a una transazione normale, il che lo rende economicamente conveniente.
Tuttavia, CoinJoin presenta alcune limitazioni, in particolare riguardo la scalabilità. È necessario che più utenti si coordinino e sincronizzino le proprie transazioni per partecipare efficacemente a CoinJoin. Questo processo di coordinazione può essere complesso e meno pratico su larga scala, rendendo CoinJoin meno adatto per situazioni in cui è richiesta una grande partecipazione.

Una terza possibile soluzione consiste nell’utilizzo delle \textbf{firme ad anello} (Riferimento spiegazione ring signature), caratteristica fondamentale di molte criptomonete basate sul protocollo CryptoNote. 

L'utilizzo di firme ad anello è vantaggioso rispetto alle altre soluzioni: Coinjoin e mixing si basano su terze parti, mentre CryptoNote permette, una volta posseduta la blockchain, di eseguire il mixing localmente. In generale le firme ad anello forniscono un livello di privacy superiore poiché il mixing è integrato nel protocollo. Dall'altra parte, sebbene un anello di dimensioni maggiori migliori l'irrintracciabilità, aumenta anche il costo di convalida della transazione, poiché la commissione è proporzionale alla dimensione della transazione, che cresce con la dimensione dell'anello.

\section{Scelta dei mixin nel protocollo Cryptonote}
Supponiamo che Alice voglia trasferire criptomonete a Bob. Essa dovrà utilizzare i suoi output di transazione (TXO) e le rispettive chiavi private per creare una nuova transazione. Per rendere la transazione irrintracciabile, Alice dovrà selezionare \(l-1\) TXO di depistaggio, detti decoy, dalla blockchain per ogni TXO reale che vuole trasferire. Dovrà poi associare ogni TXO reale a \(l-1\) depistaggi, creando una firma ad anello con il gruppo \(G_i = \{g_1, g_2, \ldots, g_l\}\), i cui elementi sono chiamati \textbf{mixin}. Grazie alla firma ad anello one-time, Alice potrà firmare la transazione con le sue chiavi private, dichiarando i mixin come potenziali spese.
In particolare, poiché tutti i TXO in input di una transazione devono avere la stessa denominazione, il software client manterrà un database di decoy disponibili, indicizzati per denominazione, da cui campionare i mixin . Alice potrà regolare individualmente le sue politiche di selezione, in quanto la transazione verrà accettata dai miner solo se gli elementi di \(G_i\) esistono sulla blockchain ed essa è in possesso della chiave privata per almeno uno di essi, indipendentemente dalla distribuzione utilizzata nella selezione dei decoy. 
Seppur l'anonimizzazione è mantenuta dal fatto che i miner non passono identificare quale elemento di \(G_i\) è stato effettivamente speso in quella transazione, mantenendo anonimi gli input all'interno del gruppo di \(l\) input potenziali, è cruciale che l'utente scelga attentamente i mixin.

Il protocollo Cryptonote non fornisce una raccomandazione esplicita su come i “mixin” dovrebbero essere scelti. L'implementazione di riferimento originale di Cryptonote includeva una politica di selezione “uniforme”, che è stata adottata (almeno inizialmente) dalla maggior parte delle implementazioni, incluso Monero. Solo successivamente quest'ultimo ha cambiato algoritmo, adottando la distribuzione Gamma nella selezione dei decoy, come vedremo in seguito.

\section{Attacks}

Prima di procedere, diamo la seguente definizione.

\begin{definition}
Si definisce \textbf{età di un mixin} la differenza tra l'altezza del blocco in cui viene utilizzato come mixin e l'altezza del blocco in cui viene prodotto, ovvero minato come output di una transazione. 
\end{definition}
In altre parole, se un TXO è registrato come la N-esima transazione sulla blockchain e appare nel set di mixin della M-esima transazione sulla blockchain, allora definiamo l'età di quel mixin come \( M - N \).

Nel corso degli ultimi anni, differenti studi  \cite{ref4}-\cite{ref5} hanno mostrato che il modo in cui gli utenti selezionano i decoy delle loro transazioni può influenzare la loro privacy e quella degli altri.

Gli autori del \cite{ref4}, in cui è possibile trovare maggiori dettagli, hanno dimostrato che l'età di un TXO, al momento del trasferimento da parte del proprietario, segue una distribuzione simile alla distribuzione Gamma. Pertanto, se l'età dei decoy non segue la stessa distribuzione, una semplice analisi sull'età degli input di ciascun TXO può portare a indovinare il TXO realmente speso con una probabilità maggiore di $\frac{1}{l}$. In particolare per dimostrare ciò sono state estratte informazioni rilevanti dalla blockchain di Monero, fino al blocco 1288774 (15 aprile 2017) e sono state memorizzate in un database a grafo. In seguito è stato sviluppato un algoritmo iterativo, nel quale, ad ogni iterazione, sono stati segnalati tutti i riferimenti ai mixin che non possono rappresentare la spesa effettiva. Questo perché si era già determinato che l'output corrispondente era stato speso in una transazione diversa. Lo stesso è stato fatto per gli input reali tra ulteriori insiemi di input di transazione. Da questo è risultato che il 63\% degli input delle transazioni Monero può essere tracciato grazie a questo approccio. E' stato dimostrato inoltre come la vulnerabilità delle transazioni Monero all'analisi deduttiva varia con il numero di mixin scelti dall'utente: le transazioni con più mixin sono significativamente meno deducibili, come ci si potrebbe aspettare. Inoltre, anche tra le transazioni con lo stesso numero di mixin, le transazioni effettuate con versioni successive del software sono meno vulnerabili.

Nonostante l'analisi sia stata condotta su Monero, poiché l'attacco di deducibilità deriva dalla procedura di campionamento dei mixin, intrinseca al protocollo Cryptonote, le altre criptovalute basate su di esso condividono le stesse vulnerabilità. Ad esempio, per Bytecoin, la prima implementazione del protocollo Cryptonote, si è dedotto l'input reale per il 29\% degli input delle transazioni con 1 o più mixin. Sebbene questo tasso sia inferiore rispetto a Monero, si ipotizza che sia il risultato di un minor utilizzo di Bytecoin.

In un altro studio \cite{ref2}, gli autori hanno dimostrato che, nella maggior parte dei casi, il TXO più recente nel set di mixin è quello realmente speso. Hanno verificato la loro euristica mediante simulazione e hanno mostrato che il 99,5\% degli input delle transazioni può essere tracciato utilizzando questo semplice attacco. Nonostante dall'aprile del 2015 gli sviluppatori di Monero sono passati a una distribuzione triangolare per selezionare i TXO precedenti come decoy, utilizzando lo stesso attacco, ancora il 96\% delle transazioni può essere tracciato. 

Seguendo il suggerimento proposto dagli autori di \cite{ref4}, Monero è poi passato alla distribuzione Gamma per la selezione dei decoy. L'uso di una distribuzione migliore, cioè una distribuzione più vicina alla distribuzione dell'età dei TXO realmente spesi, può rendere alcuni attacchi meno efficaci, come quello proposto in \cite{ref2}. Tuttavia, poiché il comportamento degli utenti cambia nel tempo \cite{reference7}, è difficile trovare e seguire costantemente la distribuzione dell'età dei TXO realmente spesi e selezionare i decoy delle transazioni utilizzando esattamente la stessa distribuzione di età.

In \cite{ref5} è stato proposto un ulteriore attacco, che sfrutta i record delle transazioni precedenti sulla blockchain per ottenere ipotesi probabilistiche sui TXO effettivamente spesi in una transazione. In particolare è stata costruita una transazione malevola per ridurre la $k$-anonimità o persino de-anonimizzare le transazioni Monero, con effetti simili al caso zero mixin.

Nell'articolo \cite{refnew} viene descritto un attacco basato sull'idea che, dato che gli algoritmi di selezione dei mixin sono di tipo probabilistico, c'è la possibilità che alcuni TXO vengano involontariamente ignorati da altri utenti e non vengano utilizzati come mixin in nessuna transazione, ma piuttosto come TXO effettivamente spesi. 

Supponendo che la rete approvi soltanto transazioni con anelli di dimensione pari a $l$. Sia \( P_t(n) \) la probabilità che un nuovo TXO, \( X_N \), venga trasferito quando esattamente \( n \) nuovi TXO sono registrati sulla blockchain dopo di esso. Sia $\epsilon(n)$ la probabilità che un TXO rimanga non speso dopo $n$ TXO, la cui dipende dal comportamento degli utenti. In particolare, se una criptovaluta viene principalmente utilizzata come mezzo di scambio, dove gli utenti scambiano frequentemente la valuta per beni e servizi, allora \(\epsilon(n)\) potrebbe essere molto vicino a zero per un \( n \) sufficientemente grande. Di seguito assumiamo che \(\epsilon(n)\) abbia un valore trascurabile per grandi \( n \) al fine di rendere le equazioni più trattabili, tuttavia, gli attacchi proposti possono essere utilizzati per qualsiasi valore di \(\epsilon(n)\).  Sia \( P_f(\cdot) \) è la distribuzione che l'algoritmo di selezione del mixin utilizza per selezionare i decoy e $P_{0}(K)$ è la probabilità che un TXO specifico rimanga ignorato per $K$ transazioni, cioè non utilizzato come decoy nelle successive $K$ transazioni, dopo essere stato emesso.
Sia \( X_i \) l'i-esimo TXO sulla blockchain. Supponiamo, per ogni \( X_N \), di esaminare gli insiemi di mixin di tutte le transazioni con output in \( \{X_{N+1}, \dots, X_{N+K}\} \). Se \( X_N \) compare nell'insieme di mixin di una sola transazione, allora è molto probabile che \( X_N \) sia il vero TXO speso in quella transazione. Se \( K \) è sufficientemente grande, allora il tasso di errore diminuisce e di conseguenza l'inferenza avrà una precisione maggiore in quanto i TXO sono spesi dall'utente dopo un tempo limitato. La Proposizione 1 fornisce la precisione di tale inferenza.

\textbf{Proposizione 1:} Sia \( X_N \) un output di una transazione che viene utilizzato, per la prima volta,come mixin esattamente \( k_1 \) transazioni dopo. Se \( X_N \) non viene utilizzato come mixin in nessun'altra transazione per almeno \( k_2 \) transazioni dopo la prima volta, allora la probabilità che la prima volta che è apparso sulla blockchain sia la sua vera spesa, indicata come \( P(A|B_{k_1,k_2}) \), è data da

\[ P(A|B_{k_1,k_2}) = \frac{P_t(k_1)}{P_t(k_1) + \epsilon(k_1 + k_2) \cdot \frac{1 - (1 - P_f(k_1))^{l-1}}{1 - (1 - P_f(k_1))^{l-1}}}. \quad (6) \]

dove gli eventi \( A \) e \( B_{k_1,k_2} \) sono definiti come:

\begin{itemize}
    \item \( A \) è l'evento in cui \( X_N \) è speso nella \( k_1 \)-esima transazione dopo la sua emissione;
    \item \( B_{k_1,k_2} \) è l'evento in cui \( X_N \) non viene utilizzato in nessuna delle \( k_1 + k_2 \) transazioni dopo la sua emissione, tranne per la \( k_1 \)-esima.
\end{itemize}

Sia \( K \) il numero più piccolo per il quale \( \epsilon(K) \) è trascurabile rispetto a \( P_t(k_1) \) per tutti i \( k_1 \) minori di \( K \). Allora, per tutti i \( k_1 \) e \( k_2 \) dove \( k_1 + k_2 > K \), la probabilità in (6) è molto vicina a 1 e l'Attacco 1 avrà una precisione molto alta.

Supponendo che $\epsilon(n)$ diminuisca all'aumentare di n, la probabilità che una transazione selezionata casualmente venga tracciata quasi certamente utilizzando l'attacco, indicata con $P_{r}(Attacco 1)$, è data da:
\begin{equation}
    P_{r}(Attacco 1) \approx P_{0}(K+1).
\end{equation}
È importante notare che, nonostante gli utenti di Monero utilizzino diversi codici sorgente, ciascuno con algoritmi differenti per la selezione dei mixin, questo tipo di attacco rimane applicabile, anche se valutare l'efficacia in questo contesto risulta complesso.